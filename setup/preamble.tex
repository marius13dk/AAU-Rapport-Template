%%%%%%%%%%%%%%%%%%%%%%%%%%%%%%%%%%%%%%%%%%%%%%%%%%%%%%%%%%%%%%%%%%%%%%%%%%%%%%%%%
\usepackage[format=plain,labelfont={bf,it},textfont=it]{caption}
\usepackage[utf8]{inputenc}
\usepackage[danish]{babel} % ændres til engelsk hvis der skrives engelsk
\usepackage{csquotes}
\usepackage{ragged2e}
\usepackage{xcolor}
\usepackage{graphicx}
\usepackage{caption}
\usepackage{tikz}
\usetikzlibrary{calc,patterns,decorations.pathmorphing,decorations.markings,matrix,fadings,shadows.blur,shapes}
\usepackage{tkz-euclide}
\usepackage{hyperref}
\usepackage[framed,thmmarks]{ntheorem}
\usepackage{mathtools}
\usepackage{textcomp}
\usepackage{siunitx}
    \sisetup{output-decimal-marker = {,}}
\usepackage[inner=20mm,outer=20mm,]{geometry}
\usepackage{titlesec}
    \titleformat*{\section}{\normalfont\Large\bfseries}
    \titleformat*{\subsection}{\normalfont\large\bfseries}
    \titleformat*{\subsubsection}{\normalfont\normalsize\bfseries}
\usepackage{float}
    \let\origdoublepage\cleardoublepage
    \newcommand{\clearemptydoublepage}{%
    \clearpage
    {\pagestyle{empty}\origdoublepage}%
    }
    \let\cleardoublepage\clearemptydoublepage
\usepackage{fancyhdr}
    \fancyhf{}
    \renewcommand{\headrulewidth}{0pt} 
    \fancyhead[RE]{\small\nouppercase\leftmark} 
    \fancyhead[LO]{\small\nouppercase\rightmark} 
    \fancyfoot[LE,RO]{\thepage} 
    \raggedbottom
\usepackage{calc}
\usepackage[backend=biber,style=numeric,]{biblatex}
    \addbibresource{bib/mybib.bib}
\usepackage[danish]{varioref}
\usepackage{blkarray, bigstrut}
\usepackage{tabularx, booktabs}
\usepackage{booktabs}% http://ctan.org/pkg/booktabs
%% Mine tilføejsler
\usepackage{array}
\usepackage{mathdots}
\usepackage{yhmath}
\usepackage{cancel}
\usepackage{color}
\usepackage{array}
\usepackage{multirow}
\usepackage{gensymb}
\usepackage{tabularx}
\usepackage[textsize=tiny]{todonotes}
\usepackage{lipsum} % til dummy tekst blokke
\usepackage{amsmath}
\usepackage{amssymb}
\usepackage{subcaption}
\usepackage{stanli}
%%%%%%%%%%%%%%%%%%%%%%%%%%%%%%%%%%%%%%%%%%%%%%%%%%%%%%%%%%%%%%%%%%%%%%%%%%%%%%%%%
% ¤¤ Kapiteludssende ¤¤ %
\definecolor{aaublue}{RGB}{33,26,82}% dark blue
\definecolor{mygreen}{rgb}{0,0.6,0}
\definecolor{mygray}{rgb}{0.5,0.5,0.5}
\definecolor{mymauve}{rgb}{0.58,0,0.82}
\definecolor{numbercolor}{gray}{0.7}		% Definerer en farve til brug til kapiteludseende
\newif\ifchapternonum

\makechapterstyle{jenor}{					% Definerer kapiteludseende %frem til ...
  \renewcommand\beforechapskip{0pt}
  \renewcommand\printchaptername{}
  \renewcommand\printchapternum{}
  \renewcommand\printchapternonum{\chapternonumtrue}
  \renewcommand\chaptitlefont{\fontfamily{pbk}\fontseries{db}\fontshape{n}\fontsize{25}{35}\selectfont\raggedleft}
  \renewcommand\chapnumfont{\fontfamily{pbk}\fontseries{m}\fontshape{n}\fontsize{1in}{0in}\selectfont\color{numbercolor}}
  \renewcommand\printchaptertitle[1]{
    \noindent
    \ifchapternonum
    \begin{tabularx}{\textwidth}{X}
    {\let\\\newline\chaptitlefont ##1\par} 
    \end{tabularx}
    \par\vskip-2.5mm\hrule
    \else
    \begin{tabularx}{\textwidth}{Xl}
    {\parbox[b]{\linewidth}{\chaptitlefont ##1}} & \raisebox{-15pt}{\chapnumfont \thechapter}
   \end{tabularx}
    \par\vskip2mm\hrule
    \fi
  }
}											% ... her

\chapterstyle{jenor}						% Valg af kapiteludseende - Google 'memoir chapter styles' for alternativer

% ¤¤ Sidehoved/sidefod ¤¤ %
\makepagestyle{Uni}							% Definerer sidehoved og sidefod udseende frem til ...
\makepsmarks{Uni}{%
	\createmark{chapter}{left}{shownumber}{}{. \ }
	\createmark{section}{right}{shownumber}{}{. \ }
	\createplainmark{toc}{both}{\contentsname}
	\createplainmark{lof}{both}{\listfigurename}
	\createplainmark{lot}{both}{\listtablename}
	\createplainmark{bib}{both}{\bibname}
	\createplainmark{index}{both}{\indexname}
	\createplainmark{glossary}{both}{\glossaryname}
}
\nouppercaseheads											% Ingen Caps oenskes

\makeevenhead{Uni}{M5-4-E20}{}{\leftmark}				    % Lige siders sidehoved (\makeevenhead{Navn}{Venstre}{Center}{Hoejre})
\makeoddhead{Uni}{\rightmark}{}{Aalborg Universitet Esbjerg}% Ulige siders sidehoved (\makeoddhead{Navn}{Venstre}{Center}{Hoejre})
\makeevenfoot{Uni}{\thepage}{}{}							% Lige siders sidefod (\makeevenfoot{Navn}{Venstre}{Center}{Hoejre})
\makeoddfoot{Uni}{}{}{\thepage}								% Ulige siders sidefod (\makeoddfoot{Navn}{Venstre}{Center}{Hoejre})
\makeheadrule{Uni}{\textwidth}{0.5pt}						% Tilfoejer en streg under sidehovedets indhold
\makefootrule{Uni}{\textwidth}{0.5pt}{1mm}					% Tilfoejer en streg under sidefodens indhold

\copypagestyle{Unichap}{Uni}								% Der dannes en ny style til kapitelsider
\makeoddhead{Unichap}{}{}{}									% Sidehoved defineres som blank på kapitelsider
\makeevenhead{Unichap}{}{}{}
\makeheadrule{Unichap}{\textwidth}{0pt}
\aliaspagestyle{chapter}{Unichap}							% Den ny style vaelges til at gaelde for chapters
															% ... her
															
\pagestyle{Uni}												% Valg af sidehoved og sidefod (benyt 'plain' for ingen sidehoved/fod)

%%%% EGNE KOMMANDOER %%%%
% ¤¤ Specielle tegn ¤¤ %
\newcommand{\tabitem}{~~\llap{\textbullet}~~}
\newcommand{\dec}{^{\circ}}									% '\dec' returnerer et gradtegn (husk $$ udenfor aligns)
\newcommand{\decC}{^{\circ}\text{C}}						% '\decC' returnerer et gradtegn + 'C' (husk $$ udenfor aligns)
\newcommand{\mdot}{\cdot}										% '\m' returnerer et gangetegn



